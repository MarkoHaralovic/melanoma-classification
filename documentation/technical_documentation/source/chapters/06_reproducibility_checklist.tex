\chapter{Reproducibility Guide}
\label{ch:reproducibility}

To reproduce our results, start by cloning the repository:

\url{https://github.com/MarkoHaralovic/melanoma-classification}

Place yourself in the root directory of the cloned repository before running any commands.

\section{System setup}
\begin{itemize}
    \item \textbf{Hardware used}: Most experiments were run on 8 NVIDIA Tesla L4 GPUs. However, we also conducted single-GPU experiments, whose results are reported.
    \item \textbf{Environment setup}: Chapter 3 provides detailed instructions for setting up the Python environment using Conda or Docker, matching the configuration used during training and evaluation.
\end{itemize}

\section{Exact configurations}
\begin{itemize}
    \item As shown in Chapter 4, Listing~\ref{lst:yaml_config_file}, we provide an example YAML configuration file and instructions on how to run the project using it.
    \item The configuration file used for our best-performing model is available in the \texttt{./configs} folder and is named \texttt{best\_run.yaml}.
    \item To perform evaluation, you can use the same \texttt{best\_run.yaml} file. Simply modify it by setting \texttt{test: True} and provide the path to the checkpoint file via the \texttt{checkpoint} parameter.
\end{itemize}

\section{Expected output directory structure}
\begin{itemize}
    \item Output files will be saved under the directory specified by the \texttt{output\_dir} argument, either through the config file or the CLI.
    \item Logs are saved in a \texttt{/logs} subfolder and include:
    \begin{itemize}
        \item \texttt{training.log} – training log + summary of metrics per epoch
        \item \texttt{log.txt} – detailed training log
        \item TensorBoard logs in \texttt{events.out.tfevents.*} format
    \end{itemize}
    \item Checkpoint files are named \texttt{checkpoint\_epoch\_<NUM>.pth}.
    \item A \texttt{config.json} file stores all arguments used during the run for future reference.
\end{itemize}

\section{Setup Environment and Load Weights}
\label{sec:setup_env}

Use one of the methods described in Section~\ref{ch:env_setup} to recreate the environment.

Create a \texttt{weights} directory and download the model from one of the following links:
\begin{itemize}
    \item \url{https://huggingface.co/Mhara/melanoma_classification/tree/main/weights}
    \item \url{https://drive.google.com/drive/folders/1n0T7q_B23edFUnzJSuIJGLcjuP-XZr-_?usp=sharing}
\end{itemize}
The provided paths lead to folders containing ONNX, \texttt{.pth}, and TorchScript checkpoints. 
For inference, use the ONNX or TorchScript checkpoints. 
For training and evaluation, use the \texttt{.pth} checkpoint.


\section{Run the Code}

To run the experiment and create our best model, use the following command:

For training run:
\begin{lstlisting}
python melanoma_train.py --config "./configs/best_run.yaml"
\end{lstlisting}

To evaluate the newly created model, edit the configuration file  by setting the `test` field to `true`. Then, run:

For evaluation run:
\begin{lstlisting}
python melanoma_train.py --config "./configs/best_run.yaml" \
                         --checkpoint <PATH_TO_PTH_WEIGHTS>
\end{lstlisting}

For inference run 

\begin{lstlisting}
python predict.py --input_folder <IMAGES_DIR> \
                         --output_csv <PATH_TO_OUTPUT_CSV>
\end{lstlisting}